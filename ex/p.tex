\documentclass[conference]{IEEEtran}
\usepackage{amsmath}
\usepackage{verbatim}
\usepackage{datatool}
\usepackage{tikz}
\usetikzlibrary{arrows,shapes}
\usepackage{graphics}

\newcounter{Examplecount}
\setcounter{Examplecount}{0}
\newenvironment{example}
{% This is the begin code
    \stepcounter{Examplecount} Example \arabic{Examplecount} 
\begin{it}
    } 
    {% This is the end code
\end{it} 
}

\begin{document}

\title{A Hard Satisfiable Problem with 160 Variables}


\author{\IEEEauthorblockN{Valentin Mayer-Eichberger}
\IEEEauthorblockA{NICTA and\\University of New South Wales\\
Valentin.Mayer-Eichberger@nicta.com.au}}

% make the title area
\maketitle


\begin{abstract}
In trying to solve a hard graph colouring problem we ran into an interesting SAT formula. The encoding uses just 160
variables and defines a special case of a rectangle-free coloring of a 18x18 grid using four colors. Rectangle-free
means that the corners of every rectangle in the grid cannot have all the same colour. Such structured satisfiable
problems pose a real challenge to SAT solvers. 
\end{abstract}
\IEEEpeerreviewmaketitle

\section{Introduction}

In 2011 a blog post of
\begin{verbatim}
    blog.computationalcomplexity.org 
\end{verbatim} 
announced a reward of $289$ \$ for a solution to the problem of 4-colouring a $17\times 17$ grid such that for each
rectangle in the grid all its corners consist of at least two different colours. A solution to $16\times 16$ was known
to exist and all grids $19 \times 19$ and larger were proven to not contain such a colouring.  In 2012 Steinbach and
Posthoff presented a solution to $17\times 17$ and $18\times 18$ \cite{Steinbach12} (every solution of larger grids
generates solutions to smaller). We provide the SAT competition with an interesting encoding for this problem which is
similar to the approach they used. It will be valuable for the community to see if any SAT solver is able to solve this
hard problem within the time-out. Our own experiments show that CDCL solvers tend to spend several hours to find a
solution. By such a benchmark we might identify advantages of non-standard SAT solver techniques. 

\section{Encoding}

Naive encodings for this problem can solve grids up to $14\times 14$  almost instantly and do not put a challenge to a SAT solver.
With some advancements and symmetry breaking one can also solve $15\times 15$  and $16\times 16$. However, no direct approach seems to tackle
the hard cases of $17\times 17$ and $18\times 18$. In this section we explain the tricks that made it possible. 

We identify a special case that can be extended to a full solution. If such a solution would exist then the problem is
solved, but a negative result would not give much insight. Luckily, it turns out that the simplification does indeed
lead to solutions. 

We simplify the problem to find a two coloring. We denote the two colours as primary and secondary, and the secondary
colour represents the three other colours of the original problem. A solution to this problem can be extended to a
solution if

\begin{itemize}
    \item only the primary colour needs to be rectangle-free,
    \item $1/4$ of all positions are filled with the primary colour, 
    \item rotating the solution by 90,180, and 270 degrees will not map a position containing a primay colour onto
another. 
\end{itemize}

We can then take a solution of this problem and fill for each rotation the mapped positions of the primary colour with
one of the remaining one. Since there are no collisions and rectangle-free is preserved under rotation, we generate a
full solution. 


%\begin{scriptsize}
%\begin{verbatim*}
%bdaaadcbdbcddacbac
%dddbdbccccbaacbbaa
%aacbdabbcadcddcdab
%babcddcaadbcbadccb
%dbcdcbcaddabcdaaab
%cdcabaddaddbbcabca
%cdbbaaaacbcbaddcdd
%bacadcbddcababbcdc
%abdcccabcdadbabdda
%cbbdcdbcbadcaaabdc
%abaddcdcabbdabcacd
%bbabbcdadaccccddba
%cadcaddbbcbbcdcaba
%dcccbadcbbcadabadb
%daabcdadbccabbadcd
%dcbabbabcaddcbdacc
%ccddaccdaaaadbdbbb
%acdacbbadbdabcccbd
%\end{verbatim*}
%\end{scriptsize}

A natural choice would be to define for each  each position in the board a Boolean variable that is true if that
position contains the primary colour. We reduce the number of variables by using the restriction that each orbit wrt. to
the 90 degree rotations should have exactly one primary colour. This can be encoded in a logarithmic fashion such that
two Boolean variables identify for each orbit in which of the four half section of the grid it exists. Furthermore, we
break symmetries by forcing the upper left position to contain a primary colour. By these reductions we get a formula
that only uses 160 variables. 

\section{Benchmark}

The set contains 4 encodings of the same problems. They have been generated by shuffling the variables, literals and
order of clauses of the encoding described above.

\section*{Acknowledgement}

NICTA is funded by the Australian Government as represented by the Department of Broadband, Communications and the
Digital Economy and the Australian Research Council through the ICT Centre of Excellence program.


\bibliographystyle{IEEEtran}
\bibliography{p}

\end{document}
